\section{Conclusion}
This project intended to solve the problem of organising sheet music, a visual representation of a piece of music which is given to performers in order to understand how a piece of music is to be performed.

There are many ways in which a piece of music could be tagged or filtered, involving symbolic, historical and bibliographical data, and as such many facets of symbolic music notation need to be addressed in order to solve this problem. The representations of music in the past have hindered automatic organisation of sheet music by these many filters, as they have used images, which do not inform a computer as to what content the file contains without the use of complex image processing.

This has lead to various users avoiding digitising their sheet music in favour of physical cupboards and more traditional storage methods, as computers did not yield a better option for organising their sheet music. The alternatives provided were to use applications which allow the user to manually tag and sort their music, a process which would be long and arduous for users with large collections of music.