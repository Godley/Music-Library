\section{Evaluation}
\subsection{Project Achievements}
This project has met all primary objectives, implementing a working renderer, a metadata scanner for a given folder, the ability to search and organise pieces according to this data, and finally, the ability to expand a collection of music by searching online music catalog APIs from within the application. 

\subsection{Further Work}
The project does not yet implement specific areas of notation such as drum tablature, guitar tablature, lyrics or chord names. These areas were decided to be of less importance to the overall goal because the expected demographic is classical musicians. However, every aspect of these sections of ignored notation has been documented in the issue tracker, and will be worked upon after assessment.
Discuss secondary objectives

\subsection{Future Developments}
\subsubsection{Porting to other operating systems}
This project was intended, though not explicitly, to be cross platform, allowing for users to have the same feature set and interface whether they were using Windows, OSX or a GNU/Linux distribution. This has not been met only because of complications creating installers for other platforms, but is intended to be implemented in the future.

It is also hoped that a Raspberry Pi compatible version can be created which may involve minimising the code and optimising certain features. With this comes potential for new input mechanisms, such as the PiPiano, an add on board which allows users to input music using buttons arranged in a piano keyboard organisation(ref), and new output mechanisims, such as Sonic Pi, which was created as an educational tool to teach children how to program using music as the inspiration and final output(ref). 

\subsubsection{Musicology expansion and more complex searching}
At present the project only allows users to search data using a english query input, but it is hoped that in future, the model of metadata can be expanded to include particular musical patterns, lyrics, and chord names, and that the querying structure can be expanded to use symbolic input. This would expand the feature set to allow musicologists to use the same system for organising as they do for analysing pieces of music, and for more intuitive input forms by musicians who regularly write music.

\subsubsection{Advanced rendering and sound output options}
It is hoped that the project may be expanded so that users can pick and choose what parts and notation are rendered and outputted to sound. This would enable the system to be used as a teaching tool, allowing teachers and students to only display notation they have already learned. It would also allow the sound output generator to be used to create accompaniment parts for practice purposes.
